\documentclass[11pt,letterpaper]{article}

%%%%%%%%%%%%%%%%%%%%%%%%%%%%%%%%%%%%%%%%%%%%%%%%%%%%%%%%%%%%%%%%%%%%%%%%%
\pagestyle{plain}                                                      %%
%%%%%%%%%% EXACT 1in MARGINS %%%%%%%                                   %%
\setlength{\textwidth}{6.5in}     %%                                   %%
\setlength{\oddsidemargin}{0in}   %% (It is recommended that you       %%
\setlength{\evensidemargin}{0in}  %%  not change these parameters,     %%
\setlength{\textheight}{8.5in}    %%  at the risk of having your       %%
\setlength{\topmargin}{0in}       %%  proposal dismissed on the basis  %%
\setlength{\headheight}{0in}      %%  of incorrect formatting!!!)      %%
\setlength{\headsep}{0in}         %%                                   %%
\setlength{\footskip}{.5in}       %%                                   %%
%%%%%%%%%%%%%%%%%%%%%%%%%%%%%%%%%%%%                                   %%
\newcommand{\required}[1]{\section*{\hfil #1\hfil}}                    %%


\usepackage{tikz}
\usepackage{pgf}
\usetikzlibrary{arrows,automata}

\usepackage{titlesec}
\usepackage{amsmath}
\usepackage{amssymb}
\usepackage{graphicx}
\usepackage[rightcaption]{sidecap}
\usepackage{dsfont}
%\usepackage{nopageno}

\usepackage{hyperref}
\usepackage{times}
\usepackage{graphicx} % more modern
%\usepackage{epsfig} % less modern
\usepackage{subfigure} 
\usepackage{url}
\usepackage{epstopdf}
\usepackage{amsfonts}
%\usepackage{caption}
\usepackage{xcolor}
\usepackage{amsmath,amssymb} % define this before the lin numbering.
%\usepackage{ruler}
%\usepackage{subfigure}
% \usepackage{color}
\usepackage{bm}

\newcommand{\sto}{{\ensuremath{\scriptscriptstyle G}}}
\newcommand{\out}{{\ensuremath{\scriptscriptstyle O}}}
\newcommand{\evt}{{\ensuremath{\scriptscriptstyle F}}}
\newcommand{\bfeat}{{\ensuremath{\bm{f}}}}
\newcommand{\by}{{\ensuremath{\bm{y}}}}
\newcommand{\bW}{{\ensuremath{\bm{W}}}}
\newcommand{\bU}{{\ensuremath{\bm{U}}}}
\newcommand{\bV}{{\ensuremath{\bm{V}}}}
\newcommand{\bh}{{\ensuremath{\bm{h}}}}
\newcommand{\bhhat}{{\ensuremath{\hat{\bm{h}}}}}
\newcommand{\bx}{{\ensuremath{\bm{x}}}}
\newcommand{\bg}{{\ensuremath{\bm{g}}}}
\newcommand{\bb}{{\ensuremath{\bm{b}}}}
\newcommand{\T}{{\ensuremath{\mathrm{T}}}}
\newcommand{\diag}{{\ensuremath{\mathrm{diag}}}}
\newcommand{\arctanh}{{\ensuremath{\mathrm{arctanh}}}}


\usepackage{amsthm}
\theoremstyle{definition}
\newtheorem{definition}{Definition}[section]
\newtheorem{theorem}{Theorem}[section]
\newtheorem{corollary}{Corollary}[theorem]
\newtheorem{lemma}[theorem]{Lemma}

\newcommand{\yhat}{\hat{y}}
\newcommand{\highlight}[2]{\textcolor{#1}{#2}}

% https://tex.stackexchange.com/questions/42726/align-but-show-one-equation-number-at-the-end
\newcommand\numberthis{\addtocounter{equation}{1}\tag{\theequation}} 

\renewcommand{\vec}[1]{{\boldsymbol #1}}
\setcounter{secnumdepth}{1}
\titleformat{\section}{\normalfont\Large\bfseries}{\thesection}{1em}{}
\titleformat{\subsection}[display]{\normalfont\bfseries\large}{}{}{}
\titlespacing{\section}{0pt}{13pt}{6.5pt}
\titlespacing{\subsection}{0pt}{11pt}{5.5pt}

\let\oldbibliography\thebibliography
\renewcommand{\thebibliography}[1]{%
  \oldbibliography{#1}%
  \setlength{\itemsep}{0pt}%
}

\begin{document}

\begin{center}

{\Large \textbf{Derivation of Energy Expression Terms}}

\vspace{1em}
Michael Iuzzolino
%Department of Computer Science \\
%University of Colorado \\
%Boulder, CO 80309-0430 USA \\
%{\tt \{mozer,kazakov\}@colorado.edu}

\end{center}

\section{Energy Functions and Activation Dynamics}
An attractor net is characterized by an energy or Lyapunov function (Koiran, 1994):
\begin{equation}
E_i = -\frac{1}{2}x_i W x_i^{\T} - bx_i^{\T} + \sum_j \int_0^{x_{ij}} f^{-1}(\xi) d \xi,
\end{equation}
where $x_i$ is the network state at some iteration $i$, $j$ is an index over units in the net, and $f(\cdot)$ is the activation function. With $f \equiv \tanh$, we have:
\begin{equation}
E_i = -\frac{1}{2}x_i W x_i^{\T} - bx_i^{\T} + \sum_j x_{ij} \arctanh(x_{ij}) + \frac{1}{2} \ln(1 - x_{ij}^2)
\end{equation}

\subsection{Derivation}
The aim is to show the following:
\[
\int_0^{x_{ij}} \tanh^{-1}(\xi) d \xi \equiv  x_{ij}\arctanh(x_{ij}) + \frac{1}{2} \ln(1 - x_{ij}^2)
\]

Beginning with $\int_0^{x_{ij}} \tanh^{-1}(\xi) d \xi$, we integrate by parts, where $\int u dv = uv - \int v du$.\\

Let $u = \tanh^{-1}(\xi)$ and $dv = d \xi$. Then,
\begin{alignat*}{3}
u &= \tanh^{-1}(\xi) \qquad && v &&= \xi \\
du &= \frac{1}{1 - \xi^2} d \xi \qquad && dv &&= d\xi\\
\end{alignat*}

Following $\int u dv = uv - \int v du$,
\begin{equation}
\xi \tanh^{-1}(\xi) \bigg\rvert_0^x - \int_0^x \frac{\xi}{1-\xi^2}d \xi
\label{eq:byparts}
\end{equation}

Now we make a substitution: $u = 1-\xi^2 \Rightarrow \xi = \sqrt{1 - u}$. Then $du = -2\xi d\xi \Rightarrow d\xi = \frac{du}{-2\sqrt{1-u}}$. 

Evaluating the first term of Eq. \ref{eq:byparts} and making the above substitution into the second term of Eq. \ref{eq:byparts} (along with algebraic simplifications and readjustment of integration bounds), we achieve:
\[
x \tanh^{-1}(x)  + \frac{1}{2} \int_1^{1-x^2} \frac{1}{u} du
\]

Since $\int du / u = ln(u) + C$, 
\[
x \tanh^{-1}(x)  + \frac{1}{2} \ln(u) \bigg\rvert_1^{1-x^2} = x \tanh^{-1}(x)  + \frac{1}{2} \left[ \ln(1-x^2) - \ln(1) \right]
\]

Finally,
\[
\int_0^{x_{ij}} \tanh^{-1}(\xi) d \xi = x \tanh^{-1}(x)  + \frac{1}{2} \ln ( 1 - x^2) 
\qed
\]
\end{document}
